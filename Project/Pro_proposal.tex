\documentclass[a4paper]{article}
\usepackage{color}
\setlength{\hoffset}{-0.5in}\hoffset-0.5in
\setlength{\textwidth}{15cm}
\topmargin = 20pt
\voffset = -20pt
\addtolength{\textheight}{2cm}
\title{Forests, Fields and Tensorflow}
\author{Abigail Millward}
\date{8th of November 2017}
\begin{document}
\thispagestyle{empty}
\null\vskip0.2in%
\begin{center}
\LARGE{{\bf 
Classificaiton of complex stresses in plants using machine learning}}
\end{center}

\vspace{0.5cm}

\begin{center}
{\Large {\bf by}}\\
\mbox{} \\
{\Large {\bf Abigail Baines (CID: 00742373)}}

{\Large Supervisor : Oliver Windram, Imperial College London, o.windram@imperial.ac.uk}
\end{center}

\vspace{1cm}

\begin{center}
\large{\bf{Department of Life Sciences \\ Imperial College London \\
London SL5 7PY \\ United Kingdom}}
\end{center}


\vspace{1.5cm}

\begin{figure}[!h]
\centering
\includegraphics[scale=0.4]{IC_Crest.eps}
\end{figure}

\vspace{1.5cm}

\begin{center}
\large{\bf{Thesis submitted as part of the requirements for the award of the \\
MRes in Computational Methods in Ecology and Evolution, Imperial College London, 2017-2018}}
\end{center}

\vspace{2cm}


  \maketitle
    
  \begin{abstract}
    Hypothesis?
    
    Keywords: Digital Pathology, Random Forests, Python, Disease detection,
    Multispectral features, Machine Learning.
  \end{abstract}
  
  \section{Introduction}
    Over the recent years, the field of digital pathology has exploded; searching
    digital pathology on Web of Science returns over 3,000 hits, and 2013 onwards
    returns over 250 papers per annum. Despite this, the avenue of digital pathology
    in plants is lagging behind, and so we believe it to be a good research avenue
    to go down to expand upon current knowledge.
    
    The study of plant pathology has always recieved much attention, due to it's link with
    agricultural yield and crop productivity \cite{donatelli_modelling_2017, waller_the_2005}. 
    Despite most critical diseases of crops having been controlled \cite{wood_sustainable_1993}
    the diseases that still manage to impact upon crop production can be devastating 
    \cite{woodham-smith_the_1962, mccook_global_2006}. Pathogens can affect the host in multiple 
    ways, with signs of the disease showing on the leaves or on the fruit.

    Past methods in detecting plant disease rely on experts simply observing the plant
    \cite{singh_detection_2017}; despite the rise in the use of machine learning for classification,
    many still rely on manual detection in this field. Our aim is to create a functioning web application
    to allow for researchers to analyse their images using a Random forest approach on
    multispectral features of the image. We will use a partially supervised approach 
    to allow for recognised plant disease experts to identify diseased leaves for a training
    set. 
    
 
3 replicates 
arabidopsis/ tomato + botrytis cinerea   + - uv stress - susect at differ levels of stress and multi stresses

Machine learning - achieving random forests + multispetral features 
leaf lesions
Touch on biology - host pathogen interaction
broadly applicable


Backend:
Normalise and standardise data - iding objects - how do we group?
Process - extract a feature
post hoc tests
Classifying objects of interest


Frontend:
Web interface?
Experts defining classifiers?
How quickly?
GUI

What am I going to have to learn to do this?

Overall outcome


Look at ilastik!!
Ask Samraat about web interface



    
    
      
  \section{Materials \& Methods}
  One of the most famous equations is:
  \begin{equation}
    E = mc^2
  \end{equation}
  This equation was first proposed by Einstein in 1905 
  \cite{einstein1905does}.
  
  \bibliographystyle{plain}
  \bibliography{FirstBiblio}
\end{document}
