\documentclass[../../Paper.tex]{subfiles}
    
\begin{document}


There has been a noteworthy increase in recent years in the use of digital pathology techinques. Plant disease detection and management are important activities for both agriculture and horticulture; by utilising digital techniques to identify diseases, it allows for quicker interventions on the infected individual(s).

This study shows that is it possible for relatively simple machine learning methods to identify both complex stress treatments, and duration of infection. These techniques will allow farmers, growers and horticultural enthusiasts to begin earlier treatment, thus minimising crop loss and reduced crop quality. 

This study also sheds light on the impact that both colour-pass filters and image normalising techniques have on digital classification. These findings should benefit future research and agricultural practises alike.

\end{document}