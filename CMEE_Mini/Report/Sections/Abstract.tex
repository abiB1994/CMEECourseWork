\documentclass[../../Paper.tex]{subfiles}
    
\begin{document}

Global temperature increases as a result of climate change are occuring 
at an alarming rate, with future temperature predictions set
to become worse. Whilst the effect of temperature on biological
traits is well researched, the potential benefits of populations acclimatizing to different temperatures 
has been somewhat overlooked. The aim of this study was
to investigate the effect of temperature on biological traits,
and whether acclimatization temperatures dampened these effects.
This study also looked at respiration response data to see whether mechanistic or
phenomological models fit the data best. Linear plots of all 
response data were originally plotted; both cubic and Schoolfield
functions were also fit to the respiration response
data. Analysis of Covariance, conducted on respiration rate, feeding rate, and
energetic efficiency, highlighted a significant effect of experimental temperature
on all trait responses. Both respiration and feeding rate significantly increased, whilst energetic efficiency
decreased. acclimatization temperature had a significant impact on energetic efficiency, 
but not on feeding rate. The model fitting results yielded the predicted outcome; 
the Schoolfield model produced a better fit to the data. Overall, the results
obtained supported findings previously made in the field of thermal
ecology, with further research into phenotypic plasticity and it's effects on thermal
tolerance advised.
\vspace{1cm}

Keywords: Thermal Performance Curves, Schoolfield, Phenotypic plasticity, Climate change, Radix balthica. 


\vspace{3cm}
Word count: 3097. 
\newpage

\end{document}
   
   