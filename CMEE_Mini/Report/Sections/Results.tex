\documentclass[../../Paper.tex]{subfiles}
    
\begin{document}


Models were originally plotted on both the entire datasets and individual streams; 
preliminary results showed that there was a significant difference between 
streams in regards to the models used, and so subsequent models were plotted
on individual streams. 

\subsection*{Results}
A linear model was plotted on the respiration data first in order to estimate a value for $E$ and $B_0$, and to calculate significance. 
The linear model was plotted over the temperatures 5-30\degree C; the parameter 
outcomes where then used as the starter parameters for subsequent models. Repiration rate significantly increased with
increasing experimental temperature (ANCOVA: $F_{1,322}$ = 1027.27, p \textless  0.000001).

\subsubsection*{Model fitting - Phenomenological vs Mechanistic}



The cubic and Schoolfield models applied to stream specific data fit with a
reasonable amount of accuracy; the worst performing model (cubic, fit to stream 5.6) 
still achieved an $R^2$ value of 0.53. 

An overarching theme produced by the model
fitting was that the Schoolfield model consistently produced a better fit than
the cubic model (Table 1, Figures 1-2). Despite  this statement, both models produced results
with $R^2$ values of above 50\%, and AICs/BICs below -290, indicating either model could be
used without risking excessive data loss. Both models produced the expected pattern of a TPC
for streams 9.6, 13.2, 13.9 and 19.3 \degree C, but whilst the Schoolfield also showed the pattern
for streams 5.6, 8.1 and 14.4 \degree C, the cubic model poorly represented these
datasets despite reasonable model fit results. Even when comparing the remaining
streams, the Schoolfield model still produces a better fit due to a steeper decrease after
the temperature peak. 


\subsection*{Feeding rate}

Feeding rate was found to significantly increase with increasing
experimental temperature(ANCOVA: $F_{1,83}$ = 21.93, p \textless  0.0001), but
there was no significant effect with either acclimation temperature 
(ANCOVA: $F_{2,83}$ = 1.37, p = 0.261) or acclimation temperature on 
experimental temperature (ANCOVA: $F_{2,83}$ = 2.01, p = 0.141). 


\subsection*{Energetic efficiency}

Energetic efficiency significantly decreased with increasing experimental 
temperature (ANCOVA: $F_{1,83}$ = 14.61, p \textless  0.0005), and there was a 
significant effect of acclimation temperature on Energetic efficiency 
(ANCOVA: $F_{2,83}$ = 5.67, p \textless  0.005, Figure 6). There was no significant effect of
acclimation on the temperature effect on energetic efficiency (ANCOVA: $F_{2,83}$ = 0.78, 
p = 0.46). 




\end{document}