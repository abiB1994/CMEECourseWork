\documentclass[../../Paper.tex]{subfiles}
    
\begin{document}

The results support 4 out of 5 of the proposed hypotheses; hypothesis 5 (acclimatization
dampening the effects of experimental temperature) was not supported, most likely due to the lack of replicates, 
although this result could also support research into rejecting the Beneficial Acclimation Hypothesis (BAH; \cite{leroi_temperature_1994}).
These findings are necessary to the current growing research body into global warming effects on species; 
further research could provide an insight into the coping strategies of species to warming pressures. 

\subsection*{Respiration rate}

The results obtained from this study support hypothesis 2, that respiration rate increases with 
temperature. This result also goes towards supporting current research into the effects of global warming
on species metabolism (\cite{grigaltchik_thermal_2012,sentis_using_2012}). The underlying mechanism behind this increase in respiration is already
well known, (Arrhenius equation; $e^{−E/kT}$), but this equation only fits the data over a normal biological operating range (5-30
\degree C). Due to this, most thermal ecological data is also plotted against polynomial models.

\subsubsection*{Model fitting}

Our model fitting analysis supported the general trend in thermal ecology models; respiration rate in
response to temperature changes show a strong fit to polynomial models. 

Both the cubic (Figure 3) and the simplified 
Schoolfield (Figure 2) models fit with reasonable accuracy; as hypothesised (hypothesis 1), the simplied 
Schoolfield produced better fit results. 

This result is most likely due to the nature of the model, with it having a biologically relevant parameters, as
opposed to the cubic, phenomenological model. Despite the cubic model still fitting approximately 
55\% of the data, due to the lack of biological meaning, it increases the risk of over-fitting 
on the data. Future experiments should aim at collecting data from a broader range of experimental temperatures
to test whether the Schoolfield model still produces the best fit. 

As well as the model comparisons, the models also produced unique parameters. The parameters of particular
interest are $E_h$ and $T_{pk}$, which varied significantly between streams. As temperature increased
from 5.6\degree C to 13.2\degree C, $T_{pk}$ gradually increased from 25 - 37\degree C, with $E_h$ also increasing
from 1.8 - 4.1. As temperature then increased from 13.2 - 19.3\degree C, $T_{pk}$ and $E_h$ decreased to 32\degree C
and 2.2 respectively. These results could be explained as a trait showing phenotypic plasticity or ``flexibility'',
whereby the individual responds to the changing environment by altering it's behaviour. Due to the peak of both $E_h$
and $T_{pk}$ at 13.2, this could further imply that there is a cost to this behaviour; past a certain 
temperature, this heat tolerance becomes more costly then beneficial.

Our species of interest, \textit{R balthica}, is already known to display flexibility in it's 
phenotypes (\cite{ahlgren_camouflaged_2013}). Our results support the potential for this species
to also display flexibility in its respiration/ metabolic rate, with the added potential for a trade 
off to explain the midrange peak. 

Future experiments to test whether all populations of \textit{R balthica} display this potential flexibility
could yield exciting and beneficial results, which also have the potential to aid in mitigation of global warming
pressures. 

\subsection*{Feeding rate}

Like both Rall \textit{et al} 2012, and Sentis \textit{et al} 2012, we found feeding rate to increase with
increasing experimental temperature, thus supporting hypothesis 2 (Figure 4). This finding is likely due to the
impact of temperature on metabolism; increasing metabolic rate will in turn lead to a higher demand for food, and 
subsequently a higher feeding rate. 

Despite the results supporting hypothesis 3, hypothesis 5 (acclimatization damping the effects of experimental temperature), was not 
supported by feeding results. This may be explained by a lack of replicates, but it is also possible that this result could 
support the rejection of the Benefical Acclimation Hypothesis (BAH; \cite{leroi_temperature_1994}).

Future experiments should focus on testing these hypotheses over a wider range of both experimental and acclimatization 
temperatures. These results should aid research into how species will cope with global warming, and the 
potential domino effect it may have on ecosystems. 

\subsection*{Energetic efficiency}

Hypothesis 4 (Energetic efficiency decreasing with temperature) was supported by the results (Figure 5). Previous 
studies have shown decreases both gut passage rate (\cite{mcconnachie_in_2004}) and ingestion efficiency 
(\cite{rall_temperature_2010}); a decrease in energetic efficiency could be caused by inefficient and slow digestion,
and therefore potentially result in starvation if energetic efficiency continues to decline (\cite{rall_temperature_2010,sentis_using_2012}). 

acclimatization temperature had a significant impact on overall energetic efficiency; warm acclimatization (19.3\degree C)
was significantly higher than the tepid (14.3\degree C) acclimatization (Figure 6). This result goes towards supporting the notion of "phenotypic 
flexibility" (\cite{piersma_phenotypic_2003,ahlgren_camouflaged_2013}); the individuals acclimated to warmer temperatures
may have been able to withstand higher experimental temperatures due to their prior, long term exposure. 

Despite both experimental and acclimatization temperature significantly impacting on energetic efficiency, the effect
of experimental and acclimatization temperature together on energetic efficiency returned a non-significant result,
causing a rejection of hypothesis 5. 

\end{document}
   
   