\documentclass[../../Paper.tex]{subfiles}
    
\begin{document}
Researching how individuals and populations respond to the changing climate is a global priority,
due to the ever increasing knowledge of the changing climatic future of our planet 
and future temperature predictions. Current temperatures are 
increasing at an alarming rate; the Earth's average surface temperature has risen
by about 0.8 \degree C over the last century alone (\cite{ipcc_climate_2014}),
with areas such as the Arctic experiencing surface air temperature (SAT)
increases of up to 2 \degree C (\cite{przybylak_recent_2007}). Predictions 
of future temperatures are looking just as bleak, with the IPCC noting 
that if nothing changes in respect to emissions, then by 2100, average surface
temperatures will further increase by 3\degree C (\cite{ipcc_climate_2014}).

Because of this, research into how both ecosystems and the individuals
within them will cope with this increase in temperature is undeniably 
important. Research has already shown that certain populations are 
shifting their ranges towards higher latitudes whilst others
 are simply loosing southern edges and failing to expand their northern ranges
(\cite{chen_rapid_2011, sunday_thermal-safety_2014, kerr_climate_2015}). Despite 
the potential for certain populations to track climate change along a latitude,
certain habitats restrict this movement (i.e. lakes, isolated patches of forest),
resulting in either adaptation to a changing environment, or local extinction.

Fundamentally, temperature increases will affect biological organisms by altering
their metabolic rates (\cite{gillooly_effects_2001}), which  will in turn impact 
on whole ecosystem dynamics. The effect of temperature
on metabolic rates is well researched (see: \cite{brown_toward_2004,price_metabolic_2010}), 
and there is  ever growing research into the effects that global warming may bring to species at
such a fundamental, biological level (see: \cite{clusella-trullas_climatic_2011,manciocco_global_2014,gandar_adaptive_2017}).

Despite this potentially negative impact, certain species display some plasticity within their
phenotypes (e.g. \textit{Sitta} sp: \cite{ghalambor_comparative_2002}; common killifish,
\textit{Fundulus heteroclitus}: \cite{schulte_thermal_2011};
wandering snail, \textit{Radix balthica}: \cite{ahlgren_camouflaged_2013}),
which could be beneficial in coping with changes to the local environment. The term 
``Phenotypic flexibility'', first proposed by Piersma \& Drent (2003), aims to describe
adjustments of physiological traits in response to changes in environmental conditions, 
both in the laboratory or the field. Since this term was first proposed, several papers
have looked into the effects of this, or similar phenomenons, on biological traits between populations 
(\cite{mckechnie_phenotypic_2008,marshall_warming_2011,ahlgren_camouflaged_2013}).

A paper by Marshall \& McQuaid (2011) looked into the 
benefits of increasing temperature on \textit{Echinolittorina malaccana}, a species of snail 
commonly found in ``stressful'' environments. They found that the Universal Temperature 
Dependence model (UTD) from the Metabolic Theory of Ecology does not explain metabolic rates 
of organisms from stressful environments, who are constantly exposed to fluctuating or high 
temperatures; their results showed a lowering of metabolism when heated for 7-42 days 
(\cite{marshall_warming_2011}). Certain populations of a species from a more ``stressful'' 
environment therefore may display unexpected outcomes in relation to the MTE; their 
population may have a higher degree of phenotypic plasticity, allowing them to cope with 
higher temperatures. Individuals  displaying this flexibility would be 
greatly beneficial to the species as a whole, and could potentially alleviate the 
pressures of global warming on the entire ecosystem. 

Several methods are currently being utilized to study individual and species specific responses
to varying temperatures, ranging from natural to lab based experiments (\cite{schulte_thermal_2011,brusch_turn_2016}).
This paper will take advantage of a naturally occurring experiment; \textit{Radix balthica}, a common
species of freshwater snail, can be found in most streams of the Hengill geothermal valley, which
displays streams varying in temperatures from 4-50 \degree C(
\cite{friberg_relationships_2009,ogorman_chapter_2012}). Stream 15 (Mean summer temperature, 50\degree C) was not included in this study
due to \textit{Radix balthica} not being present. These streams are very similar in their physical and chemical 
composition (\cite{ogorman_chapter_2012}), with temperature being the only strong 
ecologically meaningful difference among them (\cite{friberg_relationships_2009}). This provides a rare opportunity
to compare individual responses to changing temperatures across the streams, to see if 
acclimation to higher temperatures has an impact on biological traits. 

This study was undertaken to examine the effects of temperature on \textit{Radix balthica}. The data
was collected from the Hengill valley, Southwest Iceland (64\degree03'N: 21\degree18'W) by Dr E. 
O'Gorman and Dr R. Kordas. This study aims to address five questions: Does increasing experimental 
temperatures impact on respiration rates (question 1), feeding rates (question 2) and energetic 
efficiency (question 3), and, does the initial stream temperature impact on these results (question 4)?
Finally, can different mathematical models reliably fit and predict the data (question 5)?  

Based on the principles of the Metabolic Theory of Ecology and of species Thermal Performance Curves (TPCs),
I predicted respiration rate and feeding rate should both increase with increasing temperature (maximum temperature 30\degree C, hypotheses 1 and 2:
\cite{gillooly_effects_2001}), with respiration rate increasing at a faster rate than feeding rate. 
Energetic efficiency should decrease with increasing temperature (maximum temperature 30\degree C, hypothesis 3: \cite{savage_effects_2004}), as
respiration rate is predicted to increase faster than feeding rate. I predicted that the initial stream temperature
(hereafter, acclimation temperature) should aid in damping the effects of temperature on the biological traits (hypothesis 4: \cite{marshall_warming_2011}).
Finally, I predicted that the Schoolfield (mechanistic) model should produce better fit results in comparison to
a cubic (phenomenological) model (hypothesis 5) , due to the more biologically informative nature of the parameters.



%Mention comparing species response to temp between streams, mention Radix bathica - phenotypic
%plastic already known about it, oxygen consumpt - prxoy resp, feeding.
%Then mention aim of paper


%Despite high tolerance, increase temp will decrease safety margin, leading to
%more prey getting caught, resulting in dramatic changes to ecosyt. Also,
%global warming is increase. extreme conditions, and if organ. already stressed
%by coping with the excess temp, then less likely to cope with extreme. 

%TPCs are key to understand potent impact of climate change, by model temp against 
%a biological trait. ment different ways of calc relationship, all under branch
%of MTE. Ment we will use ..... 




\end{document}
   
   