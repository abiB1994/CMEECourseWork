\documentclass[../Paper.tex]{subfiles}
    
\begin{document}
How individuals and populations respond to the changing climate is a global priority,
due to the ever increasing research into the changing future of our planet 
and future temperature predictions. Current temperatures are 
increasing at an alarming rate; the Earth's average surface temperature has risen
by about 0.8\degree C over the last century alone (\cite{ipcc_climate_2014}),
with areas such as the Artic experiencing surface air temperature (SAT)
increases of up to 2\degree C (\cite{przybylak_recent_2007}). Predictions 
of future temperatures are looking just as bleak, with the IPCC noting 
that if nothing changes in respect to emissions, then by 2100, average surface
temperatures will further increase by 3\degree C (\cite{ipcc_climate_2014}).

Because of this, research into how both ecosystems and the individuals
within them will cope with this increase in temperature is undeniably 
important. Research has already shown that certain populations are 
shifting their ranges towards higher lattitudes whilst others
 are simply loosing southern edges and failing to expand their northern ranges
(\cite{chen_rapid_2011, sunday_thermal-safety_2014, kerr_climate_2015}).
Research into ectotherms climatic tolerance nearly always invokes the concept
of the ``thermal-safety margin"





\end{document}
   
   